%%%%%%%%%%%%%%%%%
% This is an example CV created using altacv.cls (v1.1.5, 1 December 2018) written by
% LianTze Lim (liantze@gmail.com), based on the
% Cv created by BusinessInsider at http://www.businessinsider.my/a-sample-resume-for-marissa-mayer-2016-7/?r=US&IR=T
%
%% It may be distributed and/or modified under the
%% conditions of the LaTeX Project Public License, either version 1.3
%% of this license or (at your option) any later version.
%% The latest version of this license is in
%%    http://www.latex-project.org/lppl.txt
%% and version 1.3 or later is part of all distributions of LaTeX
%% version 2003/12/01 or later.
%%%%%%%%%%%%%%%%

%% If you are using \orcid or academicons
%% icons, make sure you have the academicons
%% option here, and compile with XeLaTeX
%% or LuaLaTeX.
% \documentclass[10pt,a4paper,academicons]{altacv}

%% Use the "normalphoto" option if you want a normal photo instead of cropped to a circle
% \documentclass[10pt,a4paper,normalphoto]{altacv}

\documentclass[9pt,a4paper,ragged2e]{altacv}

%% AltaCV uses the fontawesome and academicon fonts
%% and packages.
%% See texdoc.net/pkg/fontawecome and http://texdoc.net/pkg/academicons for full list of symbols. You MUST compile with XeLaTeX or LuaLaTeX if you want to use academicons.

% Change the page layout if you need to
\geometry{left=1cm,right=11cm,marginparwidth=8.6cm,marginparsep=1.2cm,top=1.25cm,bottom=1.25cm}

% Change the font if you want to, depending on whether
% you're using pdflatex or xelatex/lualatex
\ifxetexorluatex
  % If using xelatex or lualatex:
  \setmainfont{Carlito}
\else
  % If using pdflatex:
  \usepackage[utf8]{inputenc}
  \usepackage[T1]{fontenc}
  \usepackage[default]{lato}
\fi

% Change the colours if you want to
\definecolor{VividPurple}{HTML}{000000}
\definecolor{SlateGrey}{HTML}{2E2E2E}
\definecolor{LightGrey}{HTML}{2E2E2E}
\colorlet{heading}{VividPurple}
\colorlet{accent}{VividPurple}
\colorlet{emphasis}{SlateGrey}
\colorlet{body}{LightGrey}
\usepackage{hyperref}

% Change the bullets for itemize and rating marker
% for \cvskill if you want to
\renewcommand{\itemmarker}{{\small\textbullet}}
\renewcommand{\ratingmarker}{\faCircle}

%% sample.bib contains your publications
\addbibresource{sample.bib}



%%% Работа с русским языком % для pdfLatex
\ifxetexorluatex
\else
\usepackage{cmap}					% поиск в~PDF
\usepackage{mathtext} 				% русские буквы в~фомулах
\usepackage[T2A]{fontenc}			% кодировка
\usepackage[utf8]{inputenc}			% кодировка исходного текста
\usepackage[english,russian]{babel}	% локализация и переносы
\usepackage{indentfirst} 			% отступ 1 абзаца
\usepackage{gensymb}				% мат символы?
\fi

\begin{document}
% Cropped to square from https://en.wikipedia.org/wiki/Marissa_Mayer#/media/File:Marissa_Mayer_May_2014_(cropped).jpg, CC-BY 2.0
%\photo{3.3cm}{profile.jpg}
\personalinfo{%
  % Not all of these are required!
  % You can add your own with \printinfo{symbol}{detail}
  \email{1alekseik1@gmail.com}
  \phone{+7 (987) 189-25-65}
%  \mailaddress{Address, Street, 00000 County}
  %\location{}
%  \homepage{marissamayr.tumblr.com/}
%  \twitter{@marissamayer}
  %\linkedin{linkedin.com/in/rishabh-jain-1605b6116/}
   \github{github.com/alekseik1} % I'm just making this up though.
%   \orcid{orcid.org/0000-0000-0000-0000} % Obviously making this up too. If you want to use this field (and also other academicons symbols), add "academicons" option to \documentclass{altacv}
}

\name{Алексей Кожарин}
\tagline{}

%% Make the header extend all the way to the right, if you want.
\begin{fullwidth}
\makecvheader
\end{fullwidth}

%% Depending on your tastes, you may want to make fonts of itemize environments slightly smaller
%\AtBeginEnvironment{itemize}{\small}

%% Provide the file name containing the sidebar contents as an optional parameter to \cvsection.
%% You can always just use \marginpar{...} if you do
%% not need to align the top of the contents to any
%% \cvsection title in the "main" bar.
\cvsection[page1sidebar]{Опыт работы}
    \begin{samepage}
        \cvevent{R\&D разработчик}{Яндекс Беспилотные автомобили}{Окт 23 -- сейчас}{Москва}
\begin{itemize}
    \item Обучил модель классификации тихоходов на трассе, настроил регулярный процесс дообучения.
    \item Писал диффузионную модель для генерации траекторий автомобилей.
\end{itemize}

        \divider

        \cvevent{Разработчик}{Яндекс}{Июнь 22 -- Окт 23}{Москва}
\begin{itemize}
    \item Разрабатывал и оптимизировал узкие места веб-сервера с 1.5к+ RPS.
    \item Дизайнил новую архитектуру сервиса регистрации доменов.
    \item Оптимизировал хранилище БД: индексы, упрощение логики.
    Это улучшило метрики качества для ключевых клиентов.
    \item Менторил стажёра, по итогам работы его были готовы взять в штат.
\end{itemize}

        \divider

        \cvevent{Младший аналитик}{Райффайзен Банк (отдел анализа данных, дирекция обслуживания корпоративных клиентов)}{Фев 21 -- сейчас}{Москва}
\begin{itemize}
    \item Работал с Apache Spark, Hadoop, Hive, Airflow
    \item Реализовал модель оттока клиентов, настроил ее регулярное обновление и оформил результаты в виде дэшбордов в Tableau
    \item Оптимизировал код по предодобрению кредитных лимитов корпоративным клиентам
    \item Вел и обучал стажера (разработка ПО)
\end{itemize}

        \divider

        % \cvevent{Младший разработчик}{Mail.ru (технический департамент)}{Май 20 -- Дек 20}{Москва}
Разрабатывал пайплайны данных на Airflow, создавал тестовое окружение и автоматизацию тестирования через Gitlab CI.

        % \divider

        % \cvevent{Младший разработчик}{Mail.ru (maps.me)}{Ноябрь 19 -- Май 20}{Москва}
\begin{itemize}
	\item Работал с чистым Python
    % \item Develop scripts for log aggregation using internal toolchain
    \item Разрабатывал скрипты аггрегации логов и подсчета статистики. Использовались инструменты внутренней разработки.
\end{itemize}

        % \divider

        % \cvevent{Стажировка <<Amgen Scholars>>}{ETH Zurich}{Июль 19 -- Август 19}{Цюрих, Швейцария}
\begin{itemize}
    \item Работал с Keras и CNN (Convolutional Neural Networks)
    \item Реализовал предложенную архитектуру CNN в Keras
    \item Адаптировал Keras API для решения задачи обучения на слабо размеченных данных (Weakly supervised learning)
\end{itemize}

    \end{samepage}
\cvsection{Образование}

\cvevent{Yandex School of Data Analysis}{Yandex}{Aug 2022 -- Sep 2023}{Moscow}
\begin{itemize}
    \item \underline{Relevant subjects}: Algorithms, Machine Learning, Deep Learning, Reinforcement Learning, Golang, C++, AB testing
    \item \underline{Average grade}: 4.6/5.
%\smallskip
\end{itemize}


\divider

%!TEX root = main.tex
\cvevent{Магистр Прикладной Математики и Физики (Физтех-школа Прикладной Математики и Информатики)}{Московский Физико-Технический Институт}{Август 2020 -- Июль 2022}{Москва}
% \begin{itemize}
% \item Релевантные предметы: \\
	% \underline{Mathematics:} calculus, analytic geometry, linear algebra, differential equations, complex analysis, probability theory\\
	% \underline{Physics:} general physics, analytical mechanics, field theory, quantum mechanics\\
%	\underline{Chemistry:} organic and inorganic, analytical chemistry, physical chemistry\\
	% \underline{Computer Science}: data structures, graph theory and algorithms, data processing in experimental physics \\
% \item Average grade: 9.11/10.00. In top 5\% of the department.
%\smallskip
% \end{itemize}

\divider

\cvevent{Bachelor in Applied Mathematics and Physics (Department of Molecular and Chemical Physics)}{Moscow Institute of Physics and Technology}{August 2016 -- June 2020}{Moscow}
\begin{itemize}
    \item \underline{Relevant subjects}: Calculus, Linear Algebra, Probability Theory, Data Structures, Graph Theory and Algorithms
    \item \underline{Average grade}: 5/5. In top 5\% of the department.
%\smallskip
\end{itemize}

\end{document}
