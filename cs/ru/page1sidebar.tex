%!TEX root = main.tex

\cvsection{Проекты}
\begin{itemize}
    \item Преподаю в блоке Python на \href{https://karpov.courses/ml-start}{курсе <<Start ML>>} от Karpov.Courses
    % \item \href{https://github.com/alekseik1/ifg-py}{Свойства идеального Ферми-газа} -- pip-модуль для численного подсчета свойств идеального Ферми-газа
    % \item \href{https://github.com/alekseik1/2018-FS-11-Frontend-Kozharin}{Сниппеты кодов для React SPA} -- написан в рамках курса <<Технотрек>>
    % \item \href{https://github.com/alekseik1/drec_stud_site}{Сайт стиральной комнаты} -- сайт и бот ВК для стиральной комнаты общежития (бронирование, платежи и контроль доступа к комнате). Написал CI/CD пайплайн, бота VK с нуля, настроил докеризацию и оркестрацию (docker-compose) и мониторинг на Graylog.
    \item \href{https://github.com/alekseik1/advanced_python_1sem_2020}{Курс <<Advanced Python>>} -- семестровый курс по разработке Python, который разработал и читал в осеннем семестре 2020 года.
    \item \href{https://github.com/alekseik1/tt-ridesharing-backend}{Backend-сервер на Flask} -- написан в рамках курса <<Технотрек>>
\end{itemize}
Больше проектов можно найти на \href{https://github.com/alekseik1/}{GitHub}.

\cvsection{Дополнительное образование}
\begin{itemize}
    % \item <<Автоматизация пайплайнов, воспроизводимость и MLOps>> от \href{https://ml-repa.ru/reproducibility-pipelines-automation-mlops}{\underline{MLRepa}}
    \item <<Публичные выступления>>: внутренний курс Яндекса от Романа Назарова
	\item <<Технотрек>>: курс по full-stack разработке от \href{https://track.mail.ru/}{\underline{Mail.ru}}
%\smallskip
\end{itemize}

\cvsection{Умения и навыки}
\begin{itemize}
	\item Языки программирования: \\
        \textbf{Python}, \textbf{C++}, \textbf{Golang}
%	\item Basic routine chemical methods: titration, pH measurement; electrical signal observing methods (CRO/DSO).
    \item SQL:
        \begin{itemize}
            \item \textbf{Postgres}: использовал в Яндексе и Райффайзен банке.
            Искал и ускорял индексами узкие места, выбирал структуру таблиц
            и работал с шардированными инсталляциями.
            % \item \textbf{MySQL}: использовал в некоторых проектах для прототипирования;
            \item \textbf{Clickhouse}: использовал в Mail.ru и Яндексе для оффлайн аналитики.
            \item \textbf{Hive}: активно использовал в DWH Райффайзен банка для аналитики в больших данных.
        \end{itemize}
    % Некоторые проекты: \href{https://github.com/alekseik1/phys\_labs}{\underline{анализ физических опытов}} и \href{https://github.com/alekseik1/machine\_learning\_coursera}{\underline{учебный репозиторий по машинному обучению}}
    % \item Frontend development: experience with React+Redux, basics of HTML, CSS.
    \item Backend на Python:
        \begin{itemize}
            \item \textbf{Flask}, \textbf{FastAPI}: использовал оба в связке с ORM.
            \item \textbf{SQLAlchemy, GINO}: основные ORM;
            \item \textbf{OpenAPI}, \textbf{sphinx}, \textbf{readthedocs}, \textbf{GitHub Pages}: использовал в коммерческой разработке и личных проектах;
            \item \textbf{asyncio}, \textbf{aiohttp}, \textbf{aioredis}, \textbf{asyncpg}: для оптимизации I/O bound задач через асинхронное выполнение;
            \item \textbf{Airflow}: писал пайплайны обработки и доставки данных на Spark;
            \item Научный стек Python: \textit{numpy}, \textit{matplotlib}, \textit{scipy}, \textit{sklearn}, \textit{pandas}, \textit{sympy}, \textit{jupyter}.
            \item \textbf{pyspark}: работал с Apache Spark через Python-обертку;
            \item \textbf{poetry, black, flake8}: для чистого кода;
            % \item \textbf{black + docformatter + isort + flake8}: для соотвествия кода стандартам PEP;
        \end{itemize}
    \item DevOps:
        \begin{itemize}
            \item \textbf{docker}, \textbf{docker-compose}: основные инструменты для деплоя;
            \item \textbf{GitHub Actions, Gitlab CI}: использовал в Райффайзен банке и личных проектах;
        \end{itemize}
\end{itemize}

% \cvachievement{\faTrophy}{}{Received accolades at Atos for Best Performance in team.}
% \cvachievement{\faTrophy}{}{Received Best Debut Award at Atos. }
% %\divider
% \cvachievement{\faInstitution}{}{Won 2nd Consolation Prize for paper presented on Cognitive Radio Networks.}
% %\divider
% \cvachievement{\faGraduationCap}{}{Got Selected in "Exclusive Scholar Program" during undergrad.}
% %\divider
% \cvachievement{\faDollar}{}{Awarded with Narotam Sekhsaria Foundation Scholarship}
%\cvsection{Strengths}

%\cvtag{Hard-working (18/24)}
%\cvtag{Persuasive}
%\cvtag{Motivator \& Leader}

%\divider\smallskip

%\cvtag{UX}
%\cvtag{Mobile Devices \& Applications}
%\cvtag{Product Management \& Marketing}


%\divider

%\cvevent{B.S.\ in Symbolic Systems}{Stanford University}{Sept 1993 -- June 1997}{}

\cvsection{Владение языками}
\begin{itemize}
	\item Английский -- Upper-Intermediate
    \item Немецкий -- Elementary
\end{itemize}

%\cvsection{PUBLICATIONS}
%\smallskip
%\begin{itemize}
%\item R. Jain, H. Tulsani, A. Bansal, "A two- tier steganographic model based on (2,2)VCS and integer wavelet transform", in Preceding of The 5th International Conference on Computing for Sustainable Global Development organized by IEEE, pp. 4731-4734, (2018).
%\smallskip
%\end{itemize}
