%!TEX root = main.tex

\cvsection{Дополнительное образование}
\begin{itemize}
	\item <<Обучение с учителем>> на \href{https://www.coursera.org/learn/supervised-learning?specialization=machine-learning-data-analysis}{\underline{Coursera}} (курс от МФТИ и Яндекс)
	\item <<Поиск структуры в данных>> на \href{https://www.coursera.org/learn/unsupervised-learning?specialization=machine-learning-data-analysis}{\underline{Coursera}} (курс от МФТИ и Яндекс)
	\item <<Теория игр>> на \href{https://www.coursera.org/learn/gametheory}{\underline{Coursera}} (курс от МФТИ и Яндекс)
	% \item "Basics of HTML and CSS" on \href{https://www.coursera.org/learn/snovy-html-i-css?specialization=razrabotka-interfeysov}{\underline{Coursera}} (MIPT and Yandex)
	% \item "Basics of JavaScript" on \href{https://www.coursera.org/learn/javascript-osnovy-i-funktsii}{\underline{Coursera}} (MIPT and Yandex)
	\item <<Технотрек>>: курс по full-stack разработке от \href{https://track.mail.ru/}{\underline{Mail.ru}}
%\smallskip
\end{itemize}


\cvsection{Проекты}
\begin{itemize}
    \item \href{https://github.com/alekseik1/ifg-py}{Свойства идеального Ферми-газа} -- pip-модуль для численного подсчета свойств идеального Ферми-газа
    \item \href{https://github.com/alekseik1/tt-ridesharing-backend}{Backend-сервер на Flask} -- написан в рамках курса <<Технотрек>>
    \item \href{https://github.com/alekseik1/2018-FS-11-Frontend-Kozharin}{Сниппеты кодов для React SPA} -- написан в рамках курса <<Технотрек>>
\end{itemize}
Больше проектов можно найти на \href{https://github.com/alekseik1/}{GitHub}.

\cvsection{Умения и навыки}
\begin{itemize}
	\item Языки программирования: \\
        \textbf{Python}, \textbf{C++}, \textbf{JavaScript}
%	\item Basic routine chemical methods: titration, pH measurement; electrical signal observing methods (CRO/DSO).
    \item SQL:
        \begin{itemize}
            \item \textbf{Postgres}: прошел курс <<Проектирование БД>> как часть программы <<Технотрек>>;
            \item \textbf{MySQL}: использовал в некоторых проектах для прототипирования;
            \item \textbf{Clickhouse}: активно использовал во время работы в Mail.ru.
        \end{itemize}
    \item Научный стек Python: \textit{numpy}, \textit{matplotlib}, \textit{scipy}, \textit{sklearn}, \textit{pandas}, \textit{sympy}, \textit{jupyter}.
    Некоторые проекты: \href{https://github.com/alekseik1/phys\_labs}{\underline{анализ физических опытов}} и \href{https://github.com/alekseik1/machine\_learning\_coursera}{\underline{учебный репозиторий по машинному обучению}}
    % \item Frontend development: experience with React+Redux, basics of HTML, CSS.
    \item Разработка backend на Python:
        \begin{itemize}
            \item \textbf{Flask}, \textbf{FastAPI}: основные фреймворки;
            \item \textbf{SQLAlchemy}: ORM для Flask;
            \item \textbf{OpenAPI}, \textbf{sphinx}: контроль документации;
            \item \textbf{asyncio}: для многопоточных программ;
            \item \textbf{Airflow}: настраивал мониторинг данных.
        \end{itemize}
    \item DevOps:
        \begin{itemize}
            \item \textbf{docker}, \textbf{docker-compose}: основные инструменты для деплоя;
            \item \textbf{Travis CI}: активно использую в личных проектах (вместе с \textbf{GitHub Actions});
            \item \textbf{Gitlab CI} использовал в Mail.ru.
        \end{itemize}
\end{itemize}

% \cvachievement{\faTrophy}{}{Received accolades at Atos for Best Performance in team.}
% \cvachievement{\faTrophy}{}{Received Best Debut Award at Atos. }
% %\divider
% \cvachievement{\faInstitution}{}{Won 2nd Consolation Prize for paper presented on Cognitive Radio Networks.}
% %\divider
% \cvachievement{\faGraduationCap}{}{Got Selected in "Exclusive Scholar Program" during undergrad.}
% %\divider
% \cvachievement{\faDollar}{}{Awarded with Narotam Sekhsaria Foundation Scholarship}
%\cvsection{Strengths}

%\cvtag{Hard-working (18/24)}
%\cvtag{Persuasive}
%\cvtag{Motivator \& Leader}

%\divider\smallskip

%\cvtag{UX}
%\cvtag{Mobile Devices \& Applications}
%\cvtag{Product Management \& Marketing}


%\divider

%\cvevent{B.S.\ in Symbolic Systems}{Stanford University}{Sept 1993 -- June 1997}{}

\cvsection{Владение языками}
\begin{itemize}
	\item Английский -- Upper-Intermediate
    \item Немецкий -- Elementary
\end{itemize}

%\cvsection{PUBLICATIONS}
%\smallskip
%\begin{itemize}
%\item R. Jain, H. Tulsani, A. Bansal, "A two- tier steganographic model based on (2,2)VCS and integer wavelet transform", in Preceding of The 5th International Conference on Computing for Sustainable Global Development organized by IEEE, pp. 4731-4734, (2018).
%\smallskip
%\end{itemize}
