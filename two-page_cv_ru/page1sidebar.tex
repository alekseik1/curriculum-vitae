%!TEX root=mmayer.tex

\cvevent{Младший разработчик}{Mail.ru (технический департамент)}{Май 20 -- сейчас}{Москва}
\begin{itemize}
    \item Работаю с ClickHouse, MySQL. Разрабатываю ПО для внутреннего использования (Python), тестирую в docker-compose с использованием копии боевого окружения.
\end{itemize}

\cvsection{Дополнительное образование}
\begin{itemize}
	\item <<Обучение на размеченных данных>> на \href{https://www.coursera.org/learn/supervised-learning?specialization=machine-learning-data-analysis}{\underline{Coursera}} (МФТИ и Яндекс)
	\item <<Поиск структуры в данных>> на \href{https://www.coursera.org/learn/unsupervised-learning?specialization=machine-learning-data-analysis}{\underline{Coursera}} (МФТИ и Яндекс)
	\item <<Теория игр>> на \href{https://www.coursera.org/learn/gametheory}{\underline{Coursera}} (МФТИ)
	\item <<Основы HTML и CSS>> на \href{https://www.coursera.org/learn/snovy-html-i-css?specialization=razrabotka-interfeysov}{\underline{Coursera}} (МФТИ и Яндекс)
	\item <<Основы JavaScript>> на \href{https://www.coursera.org/learn/javascript-osnovy-i-funktsii}{\underline{Coursera}} (МФТИ и Яндекс)
	\item Курс full-stack разработки <<Технотрек>> от \href{https://track.mail.ru/}{\underline{Mail.ru}}
%\smallskip
\end{itemize}


\cvsection{Основные проекты}
\begin{itemize}
    \item \href{https://github.com/alekseik1/ifg-py}{\underline{Свойства идеального Ферми-газа}} -- pip-модуль для численного подсчета свойств Ферми-газа
    \item \href{https://github.com/alekseik1/tt-ridesharing-backend}{Backend сервер на Flask} -- написан в рамках программы <<Технотрек>>
    \item \href{https://github.com/alekseik1/2018-FS-11-Frontend-Kozharin}{Наброски SPA-месседжера на React} -- написан в рамках программы <<Технотрек>>
\end{itemize}



\cvsection{Навыки}
\begin{itemize}
	\item Языки программирования: \\ 
	Python, C++, JavaScript
	\item Linux и MacOS терминал (Bash, Zsh)
%	\item Basic routine chemical methods: titration, pH measurement; electrical signal observing methods (CRO/DSO).
	\item Научный стек Python:
	numpy, matplotlib, scipy, sklearn, pandas, sympy, jupyter. Некоторые проекты на GitHub: \href{https://github.com/alekseik1/phys\_labs}{\underline{анализ физ. экспериментов}} и \href{https://github.com/alekseik1/machine\_learning\_coursera}{\underline{учебный репо по курсам машинного обучения}}
	\item LaTeX. Некоторые проекты на GitHub: \href{https://github.com/alekseik1/quest\_on\_choice}{\underline{заметки по физическим явлениям}} и \href{https://github.com/alekseik1/cpp\_mipt\_study}{\underline{конспекты, лекции и задачи по C++}}
    \item Docker и docker-compose
    \item PostgreSQL/MySQL: использовал SQL в некоторых проектах, прежде чем перейти на ORM (SQLAlchemy). Работал с диалектом ClickHouse
    \item OpenAPI (Swagger) -- использовал для согласования backend API с фронтендом во время работы над выпускным проектом в <<Технотреке>>
\end{itemize}

% \cvachievement{\faTrophy}{}{Received accolades at Atos for Best Performance in team.}
% \cvachievement{\faTrophy}{}{Received Best Debut Award at Atos. }
% %\divider
% \cvachievement{\faInstitution}{}{Won 2nd Consolation Prize for paper presented on Cognitive Radio Networks.}
% %\divider
% \cvachievement{\faGraduationCap}{}{Got Selected in "Exclusive Scholar Program" during undergrad.}
% %\divider
% \cvachievement{\faDollar}{}{Awarded with Narotam Sekhsaria Foundation Scholarship}
%\cvsection{Strengths}

%\cvtag{Hard-working (18/24)} 
%\cvtag{Persuasive}
%\cvtag{Motivator \& Leader}

%\divider\smallskip

%\cvtag{UX}
%\cvtag{Mobile Devices \& Applications}
%\cvtag{Product Management \& Marketing}


%\divider

%\cvevent{B.S.\ in Symbolic Systems}{Stanford University}{Sept 1993 -- June 1997}{}

\cvsection{Языки}
\begin{itemize}
	\item Английский --- Upper-Intermediate
    \item Немецкий -- Elementary
\end{itemize}

%\cvsection{PUBLICATIONS}
%\smallskip
%\begin{itemize}
%\item R. Jain, H. Tulsani, A. Bansal, "A two- tier steganographic model based on (2,2)VCS and integer wavelet transform", in Preceding of The 5th International Conference on Computing for Sustainable Global Development organized by IEEE, pp. 4731-4734, (2018).
%\smallskip
%\end{itemize}
