%%%%%%%%%%%%%%%%%
% This is an example CV created using altacv.cls (v1.1.5, 1 December 2018) written by
% LianTze Lim (liantze@gmail.com), based on the
% Cv created by BusinessInsider at http://www.businessinsider.my/a-sample-resume-for-marissa-mayer-2016-7/?r=US&IR=T
%
%% It may be distributed and/or modified under the
%% conditions of the LaTeX Project Public License, either version 1.3
%% of this license or (at your option) any later version.
%% The latest version of this license is in
%%    http://www.latex-project.org/lppl.txt
%% and version 1.3 or later is part of all distributions of LaTeX
%% version 2003/12/01 or later.
%%%%%%%%%%%%%%%%

%% If you are using \orcid or academicons
%% icons, make sure you have the academicons
%% option here, and compile with XeLaTeX
%% or LuaLaTeX.
% \documentclass[10pt,a4paper,academicons]{altacv}

%% Use the "normalphoto" option if you want a normal photo instead of cropped to a circle
% \documentclass[10pt,a4paper,normalphoto]{altacv}

\documentclass[10pt,a4paper,ragged2e]{altacv}

%% AltaCV uses the fontawesome and academicon fonts
%% and packages.
%% See texdoc.net/pkg/fontawecome and http://texdoc.net/pkg/academicons for full list of symbols. You MUST compile with XeLaTeX or LuaLaTeX if you want to use academicons.

% Change the page layout if you need to
\geometry{left=1cm,right=11cm,marginparwidth=8.6cm,marginparsep=1.2cm,top=1.25cm,bottom=1.25cm}

% Change the font if you want to, depending on whether
% you're using pdflatex or xelatex/lualatex
\ifxetexorluatex
  % If using xelatex or lualatex:
  \setmainfont{Carlito}
\else
  % If using pdflatex:
  \usepackage[utf8]{inputenc}
  \usepackage[T1]{fontenc}
  \usepackage[default]{lato}
%%% Работа с русским языком % для pdfLatex
\usepackage{cmap}					% поиск в~PDF
\usepackage{mathtext} 				% русские буквы в~фомулах
\usepackage[T2A]{fontenc}			% кодировка
\usepackage[utf8]{inputenc}			% кодировка исходного текста
\usepackage[english,russian]{babel}	% локализация и переносы
\usepackage{indentfirst} 			% отступ 1 абзаца
\usepackage{gensymb}				% мат символы?
\fi

% Change the colours if you want to
\definecolor{VividPurple}{HTML}{000000}
\definecolor{SlateGrey}{HTML}{2E2E2E}
\definecolor{LightGrey}{HTML}{2E2E2E}
\colorlet{heading}{VividPurple}
\colorlet{accent}{VividPurple}
\colorlet{emphasis}{SlateGrey}
\colorlet{body}{LightGrey}
\usepackage{hyperref}

% Change the bullets for itemize and rating marker
% for \cvskill if you want to
\renewcommand{\itemmarker}{{\small\textbullet}}
\renewcommand{\ratingmarker}{\faCircle}

%% sample.bib contains your publications
\addbibresource{sample.bib}

\begin{document}
\name{Алексей Кожарин}
\tagline{}
% Cropped to square from https://en.wikipedia.org/wiki/Marissa_Mayer#/media/File:Marissa_Mayer_May_2014_(cropped).jpg, CC-BY 2.0
%\photo{3.3cm}{profile.jpg}
\personalinfo{%
  % Not all of these are required!
  % You can add your own with \printinfo{symbol}{detail}
  \email{1alekseik1@gmail.com}
  \phone{+7 (987) 189-25-65}
%  \mailaddress{Address, Street, 00000 County}
  %\location{}
%  \homepage{marissamayr.tumblr.com/}
%  \twitter{@marissamayer}
  %\linkedin{linkedin.com/in/rishabh-jain-1605b6116/}
   \github{github.com/alekseik1} % I'm just making this up though.
%   \orcid{orcid.org/0000-0000-0000-0000} % Obviously making this up too. If you want to use this field (and also other academicons symbols), add "academicons" option to \documentclass{altacv}
}

%% Make the header extend all the way to the right, if you want.
\begin{fullwidth}
\makecvheader
\end{fullwidth}

%% Depending on your tastes, you may want to make fonts of itemize environments slightly smaller
%\AtBeginEnvironment{itemize}{\small}

%% Provide the file name containing the sidebar contents as an optional parameter to \cvsection.
%% You can always just use \marginpar{...} if you do
%% not need to align the top of the contents to any
%% \cvsection title in the "main" bar.
\cvsection[page1sidebar]{Образование}

\cvevent{Бакалавриат \\ Факультет Молекулярной и Химической Физики}{Московский Физико-Технический Институт}{Август 2016 -- Июль 2020}{Moscow}
\begin{itemize}
\item Релевантные предметы \\
	\underline{Математика:} анализ, аналитическая геометрия, линейная алгебра, дифференциальные уравнения, ТФКП, теория вероятности\\
	% \underline{Физика:} общая физика, аналитическая механика, теория поля, квантовая механика\\
%	\underline{Chemistry:} organic and inorganic, analytical chemistry, physical chemistry\\
	\underline{Computer Science}: алгоритмы и структуры данных, теория графов\\
\item Средний балл: 9.11/10.00 (5.0 по пятибалльной шкале)
%\smallskip
\end{itemize}

%\cvsection{TECHNICAL SKILLS}
%
%\cvskill{C/C++, Digital System Design}{4}
%\cvskill{OpenCL, Makefile, Shell Scripting}{2}
%\cvskill{Verilog/System Verilog, Python}{3}

%\divider

\cvsection{Опыт}
\cvevent{Junior Python backend разработчик}{Take Wing Co}{Май 18 -- Октябрь 18}{Москва}
\begin{itemize}
    \item Работал с Bitcoin API, Celery и JSON-RPC
    \item Разработал модуль для интеграции с основной сетью Bitcoin через JSON-RPC 
    \item Прописал API для модуля, настроил взаимодействие с остальными компонентами проекта через Celery
\end{itemize}
\divider

\cvevent{Тьютор физики}{Московский Физико-Технический Институт}{Октябрь 18 -- Декабрь 19}{Москва}
\begin{itemize}
	\item Проверял домашнее задание студентов и организовывал тесты
\end{itemize}
\divider

\cvevent{Стажировка <<Amgen Scholars>>}{ETH Zurich}{Июль 19 -- Август 19}{Цюрих, Швейцария}
\begin{itemize}
	% \item Worked with Keras and CNNs
    \item Работал с Keras и CNN (Convolutional Neural Networks)
    % \item Implemented proposed CNN architecture in Keras
    \item Реализовал предложенную архитектуру CNN в Keras
    % \item Adapted Keras API for Weakly supervised learning problem
    \item Адаптировал Keras API для решения задачи обучения на слабо размеченных данных (Weakly supervised learning)
\end{itemize}
\divider

\cvevent{Младший разработчик}{Mail.ru (maps.me)}{Ноябрь 19 -- Май 20}{Москва}
\begin{itemize}
	\item Работал с чистым Python
    % \item Develop scripts for log aggregation using internal toolchain
    \item Разрабатывал скрипты аггрегации логов и подсчета статистики. Использовались инструменты внутренней разработки
\end{itemize}



% \cvsection{Honors \& Awards}
% \cvevent{Prize-winner of the final stage}{Russian National Olympiad of Schoolchildren in Chemistry}{Apr 14 and Apr 15}{}


% \cvevent{Product Engineer}{Google}{23 June 1999 -- 2001}{Palo Alto, CA}

% \begin{itemize}
% \item Joined the company as employe \#20 and female employee \#1
% \item Developed targeted advertisement in order to use user's search queries and show them related ads
% \end{itemize}

%\cvsection{A Day of My Life}

% Adapted from @Jake's answer from http://tex.stackexchange.com/a/82729/226
% \wheelchart{outer radius}{inner radius}{
% comma-separated list of value/text width/color/detail}
% Some ad-hoc tweaking to adjust the labels so that they don't overlap
% \wheelchart{1.5cm}{0.5cm}{%
%   10/10em/accent!30/Sleeping \& dreaming about work,
%   25/9em/accent!60/Public resolving issues with Yahoo!\ investors,
%   5/13em/accent!10/\footnotesize\\[1ex]New York \& San Francisco Ballet Jawbone board member,
%   20/15em/accent!40/Spending time with family,
%   5/8em/accent!20/\footnotesize Business development for Yahoo!\ after the Verizon acquisition,
%   30/9em/accent/Showing Yahoo!\ employees that their work has meaning,
%   5/8em/accent!20/Baking cupcakes
% }

\clearpage

% \cvsection[page2sidebar]{Publications}

\nocite{*}

% \printbibliography[heading=pubtype,title={\printinfo{\faBook}{Books}},type=book]

% \divider

% \printbibliography[heading=pubtype,title={\printinfo{\faFileTextO}{Journal Articles}}, type=article]

% \divider

% \printbibliography[heading=pubtype,title={\printinfo{\faGroup}{Conference Proceedings}},type=inproceedings]

% %% If the NEXT page doesn't start with a \cvsection but you'd
% %% still like to add a sidebar, then use this command on THIS
% %% page to add it. The optional argument lets you pull up the
% %% sidebar a bit so that it looks aligned with the top of the
% %% main column.
% % \addnextpagesidebar[-1ex]{page3sidebar}


\end{document}
